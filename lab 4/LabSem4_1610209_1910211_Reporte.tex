%Plantilla HCCB para Ubuntu

\documentclass{article}

%Los paquetes

\usepackage{amsmath, amsfonts} %Paquetes para manejar cosas de matemática
\usepackage[spanish, es-tabla]{babel} %Paquete para utilizar LaTeX en español. La parte "es-tabla" es para que salga "TABLA 1" (en lugar de "CUADRO 1" que es el defecto)
\usepackage{graphicx} %Paquete para insertar imágenes
\usepackage{setspace} %Paquete para trabajar con las medidas en el documento
\usepackage[a4paper]{geometry} %Paquete para trabajar con el tamaño del papel y los márgenes

%Papel y márgenes

\geometry{top=1.5cm, bottom=1.5cm, left=1.5cm, right=1.5cm} %Márgenes
\setstretch{1} %Interlineado

\begin{document}
	\title{\normalsize{Universidad Sim\'on Bol\'ivar\\
			Departamento de Computaci\'on y Tecnolog\'ia de la Informaci\'on\\
			Laboratorio de Algoritmos y Estructuras II - CI2692}}
	\author{\normalsize{Profesor: Guillermo Palma}\\
		\normalsize{Estudiantes:}\\
		\normalsize{Hayde\'e Castillo Borgo. Carnet: 16-10209}\\
		\normalsize{Jes\'us Prieto. Carnet: 19-10211}}
	\date{\normalsize{Trimestre Abril-Julio 2023}}
	\maketitle
	\begin{center}
		\Large{\textbf{Laboratorio de la semana 4}}
	\end{center}\:
	
	En el presente laboratorio se llev\'o a cabo un estudio experimental de los algoritmos Quicksort, desarrollado en \cite{1}, Quicksort Three Way, desarrollado en \cite{2}, y Quicksort Dual Pivot, desarrollado en \cite{3}; a trav\'es de un computador Intel® Core™ i5-2450M CPU @ 2.50GHz × 4, con 8Gb de RAM y sistema operativo Ubuntu 20.04.6 LTS. Para dicho estudio se emple\'o el lenguaje de programaci\'on Kotlin en su versi\'on 1.8.21 y Java Virtual Machine JVM, versi\'on 11.0.19.\\
	
	Los tres algoritmos mencionados anteriormente son diferentes implementaciones de Quicksort, un algoritmo de ordenamiento basado en el esquema de Divide-and-Conquer; y el objetivo del presente estudio es verificar si las versiones mejoradas Three Way y Dual Pivot, en efecto son m\'as eficientes que el Quicksort cl\'asico.\\
	
	Para ello se ejecutaron las tres versiones del algoritmo sobre un conjunto de arreglos generados aleatoriamente, de la siguiente manera:\\
	
	Se consideraron cuatro tamaños: $N = 500.000$, $N = 1.000.000$, $N = 1.500.000$ y $N = 2.000.000$; y para cada $N$ se generaron $10$ arreglos, cuyos elementos ven\'ian dados por valores en el intervalo $[0,N]$. Luego, se ejecut\'o cada variante de Quicksort en cada uno de los $10$ arreglos, se tom\'o el tiempo correspondiente a cada prueba y se calcularon los promedios y desviaciones est\'andar, agrupando las pruebas de acuerdo al tamaño del arreglo.\\
	
	Dichos resultados se presentan en la siguiente tabla y en el siguiente gr\'afico. 

	$$\begin{array}{|c|ccc|}
		\hline
		\textbf{Algoritmo} & \textbf{Tamaño de los arreglos} &
		\textbf{Tiempo promedio} & \textbf{Desviación estándar}\\
		 & N & \text{(en segundos)} & \text{(en segundos)}\\
		\hline
		\text{Quicksort Cl\'asico} & 500000 & 0.2323 & 0.0352\\
		\text{Quicksort Three Way} & 500000 & 0.1652 & 0.0492\\
		\text{Quicksort Dual Pivot} & 500000 & 0.2078 & 0.0993\\
		\hline
		\text{Quicksort Cl\'asico} & 1000000 & 0.4989 & 0.0904\\
		\text{Quicksort Three Way} & 1000000 & 0.3688 & 0.1006\\
		\text{Quicksort Dual Pivot} & 1000000 & 0.4143 & 0.0517\\
		\hline
		\text{Quicksort Cl\'asico} & 1500000 & 0.9236 & 0.0771\\
		\text{Quicksort Three Way} & 1500000 & 0.6256 & 0.1004\\
		\text{Quicksort Dual Pivot} & 1500000 & 0.7525 & 0.0923\\
		\hline
		\text{Quicksort Cl\'asico} & 2000000 & 1.299 & 0.0908\\
		\text{Quicksort Three Way} & 2000000 & 0.8963 & 0.1081\\
		\text{Quicksort Dual Pivot} & 2000000 & 1.1885 & 0.1898\\
		\hline
	\end{array}$$\:
	
	De lo obtenido se concluye que, en efecto, las variantes Three Way y Dual Pivot son m\'as eficientes que el Quicksort cl\'asico. A\'un m\'as, se obtuvo que la variante Three Way result\'o ser la m\'as eficiente de las tres. 

	
	\begin{thebibliography}{}
		\bibitem{1} Cormen, T., Leirserson, C., Rivest, R., and Stein, C. \textit{Introduction to Algo-
		rithms}, 3ra ed. McGraw Hill, 2009.
		\bibitem{2} Sedgewick, R., and Bentley, J. Quicksort is optimal. https://sedgewick.
		io/wp-content/uploads/2022/03/2002QuicksortIsOptimal.pdf, 2002. KnuthFest,
		Stanford University.
		\bibitem{3} Wild, S., and Nebel, M. E. Average case analysis of java 7’s dual pivot quicksort.
		In \textit{Algorithms–ESA 2012: 20th Annual European Symposium, Ljubljana, Slovenia, Sep-
		tember 10-12, 2012. Proceedings 20} (2012), Springer, pp. 825–836.
	\end{thebibliography}
	
\end{document}